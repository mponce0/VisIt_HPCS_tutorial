\begin{comment}
  http://www.vtk.org/download - Download VTK textbook examples and data - PineRoot.tgz
  http://www.vtk.org/files/release/7.0/VTKLargeData-7.0.0.zip
  http://www.visitusers.org/index.php?title=Short_Tutorial
\end{comment}

\begin{frame}{Quantitative analysis}{}
  \begin{itemize}\setlength{\itemsep}{3mm}
    % - p.233
  \item Expressions (similar to the Calculator filter in ParaView)
  \item Pick: zone/node, spreadsheet, time curves
  \item Lineout
  \item Queries
  \item Creating new time sliders with database correlations
  \end{itemize}
\end{frame}

\begin{frame}{Expressions}
  \begin{itemize}\setlength{\itemsep}{5mm}
  \item Create new derived variables from existing ones
  \item Mathematical expressions can operate on scalars, vectors, tensors
  \item {\bf Option 1}: use Standar Editor to select existing functions and expressions
    \begin{itemize}
    \item similar to the Calculator filter in ParaView
    \end{itemize}
  \item {\bf Option 2}: use Python Expression Editor - this is an advanced option for working with VTK objects
    \begin{itemize}
    \item similar to the Programmable filter in ParaView
    \end{itemize}
  \end{itemize}
\end{frame}

\begin{frame}{Expressions}
  \begin{beamerboxesrounded}[upper=block head,lower=block body,shadow=true]{}
    \begin{itemize}\setlength{\itemsep}{0mm}
    \item Load \hi{noise.silo}, visualize \hi{hardyglobal} with Pseudocolor
    \item Controls \ra Expressions \ra New, set name = \hi{squared}, type = Scalar Mesh Variable,
      definition = \hi{hardyglobal$\hat{\,\,\,}$2}, click Apply
    \item Now in the list of variables switch to \hi{squared}
    \end{itemize}
  \end{beamerboxesrounded}
  \medskip
  \pause
  \begin{beamerboxesrounded}[upper=block head,lower=block body,shadow=true]{}
    \begin{itemize}\setlength{\itemsep}{0mm}
    \item Controls \ra Expressions \ra New, set name = \hi{gradient}, type = Vector Mesh Variable,
      definition = \hi{gradient(hardyglobal)}
    \item You can use the dropdown menus to accelerate typing (gradient
      will be found in Insert Function \ra Miscellaneous)
    \item Add a vector plot to show \hi{gradient}, picking ``Uniformly located troughout the mesh'',
      displaying 5000 vectors, making them bigger, overlaying onto a semi-transparent Pseudocolor
      plot of \hi{hardyglobal}
    \end{itemize}
  \end{beamerboxesrounded}
  \medskip
  \pause
  \begin{beamerboxesrounded}[upper=block head,lower=block body,shadow=true]{}
    \begin{itemize}\setlength{\itemsep}{0mm}
    \item Controls \ra Expressions \ra New, set name = \hi{truncated}, type = Vector Mesh Variable,
      definition = \hi{max(2,hardyglobal)}
    \item Now make a Pseudocolor plot of \hi{truncated}
    \end{itemize}
  \end{beamerboxesrounded}
\end{frame}

\begin{frame}{Pick}
  \begin{itemize}\setlength{\itemsep}{3mm}
  \item Interactively pick values inside the visualization window
  \item Load \hi{noise.silo}, visualize \hi{hardyglobal} with Pseudocolor, apply Operators \ra Selection \ra Clip,
    click Draw
  \item{\color{red} Data here is defined on nodes, not zones}
    \pause
  \item Right click on the visualization, select Mode \ra Zone Pick (or use a mode button in the vis
        window
    toolbar), and click anywhere on the vis -- it'll display 8 nodes forming the zone, and their variable
    values
    \begin{itemize}\setlength{\itemsep}{0mm}
    \item each pick point leaves a marker that you can look up in the Pick window
    \item the Pick window displays information in tabs arranged by a point
    \end{itemize}
    \pause
  \item Similarly, Mode \ra Node Pick will display a single node, its variable, and its 8 ``incident''
    zones
  \end{itemize}
\end{frame}

\begin{frame}{Pick}
  \begin{itemize}\setlength{\itemsep}{3mm}
  \item Selecting Mode \ra Spreadsheet Pick shows a spreadsheet view of one of the dataset variables
    highlighting the picked node (i,j,k) and its value of the variable (can't close the spreadsheet
    window while in this mode!)
    \pause
  \item Mode \ra Navigate will take you back to default interaction
    \pause
  \item Try loading a time-dependent dataset, e.g., \texttt{2d0**.vtk}
    \begin{itemize}\setlength{\itemsep}{0mm}
    \item a time slider will become active
    \item depending on the data use either Zone Pick or Node Pick
    \item inside the Pick window select ``Create time curve with next pick''
    \end{itemize}
  \end{itemize}
\end{frame}

\begin{frame}{Pick: lineout}
  \begin{itemize}\setlength{\itemsep}{2mm}
  \item Extracts 1D curves from 2D data (does not seem to work on 3D datasets)
  \item Load a 2D dataset or apply Operators \ra Slicing \ra Slice to a 3D dataset
  \item Select Mode \ra Lineout and draw a line -- the profile will be plotted in a new window
  \end{itemize}
\end{frame}

\begin{frame}{Queries}
  \begin{itemize}\setlength{\itemsep}{2mm}
  \item Controls \ra Query
  \item {\bf Option 1}: use Standard Queries
    \begin{itemize}\setlength{\itemsep}{0mm}
    \item very useful: Memory Usage
    \item quick ways to probe data: MinMax, NumNodes, NumZones, Average Value, Volume
    \item Lineout and Pick are also queries (this time enter selection manually)
    \item certain queries provide a Time Curve button that calculates the query on each time step and
      creates a curve
    \end{itemize}
  \item {\bf Option 2}: use Python Query Editor for custom queries -- this is an advanced topic
  \end{itemize}
\end{frame}

\begin{frame}{Database correlations}
  \begin{itemize}\setlength{\itemsep}{2mm}
  \item In VisIt each time-varying database (if more than one loaded) gets its own independent slider
  \item Sometimes it's useful to compare two time-varying databases, but one would need to set them both
    to the same moment(s) in time
  \item Controls \ra Database Correlations lets you do this with a single time slider for both databases,
    using one of four correlation methods
    \pause
  \item Can try creating a single time slider from \texttt{2d0**.vtk} and \texttt{modified0**.vtk}
    \begin{enumerate}\setlength{\itemsep}{0mm}
    \item load both databases, and for each draw Pseudocolor \ra Elevate, make them both visible
    \item verify you can animate either switching the active time slider
    \item now select Controls \ra Database Correlations, click New, use Correlation Method = Padded
      Index, select both Sources and move them to Correlated Sources, click Create Database Correlation
    \item a new active time slider appears that lets you animate both
    \end{enumerate}
  \end{itemize}
\end{frame}

\begin{frame}{Advanced topics in quantitative analysis}
  \begin{itemize}\setlength{\itemsep}{3mm}
  \item Cross-mesh field evaluation (CMFE) and database comparison
%  chapter 12 section 4.0 of usersManual.pdf
%  \url{http://www.visitusers.org/index.php?title=Cmfe_overview}
%  \url{http://www.visitusers.org/index.php?title=Cross_Mesh_Field_Evaluation}
%  The CMFE expressions evaluate a field from a donor mesh onto a target mesh to form a new field.
  \item   Data level comparison wizard
%    \url{http://www.visitusers.org/index.php?title=VisIt-tutorial-comparison}
  \end{itemize}
\end{frame}
