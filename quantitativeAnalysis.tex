\begin{frame}{Quantitative analysis}{}
  \begin{itemize}\setlength{\itemsep}{3mm}
    % - p.233
  \item Expressions (analogue of the Calculator filter in ParaView)
  \item Pick: zone/node, spreadsheet, time curves
  \item Lineout
  \item Queries
  \item Time sliders and database correlations
  \end{itemize}
\end{frame}

\begin{frame}{Expressions}
  \begin{itemize}\setlength{\itemsep}{3mm}
  \item Create new derived variables from existing ones
  \item Mathematical expressions can operate on scalars, vectors, tensors
  \item {\bf Option 1}: use Standar Editor to select existing functions and expressions \\\ding{48}
    analogous to the Calculator filter in ParaView
  \item {\bf Option 2}: use Python Expression Editor - this is an advanced option for working
    with VTK objects; we'll cover it in the Scripting session \\\ding{48} analogous to the Programmable
    filter in ParaView
  \end{itemize}
\end{frame}

\begin{frame}{Expressions}
  \begin{beamerboxesrounded}[upper=block head,lower=block body,shadow=true]{}
    \begin{itemize}\setlength{\itemsep}{0mm}
    \item Load sineEnvelope.nc, visualize density with Pseudocolor
    \item Controls \ra Expressions \ra New, set name = densitySquared, type = Scalar Mesh Variable,
      definition = density$\hat{\,\,\,}$2, click Apply
    \item Now in the list of variables switch to densitySquared
    \end{itemize}
  \end{beamerboxesrounded}
  \pause
  \begin{beamerboxesrounded}[upper=block head,lower=block body,shadow=true]{}
    \begin{itemize}\setlength{\itemsep}{0mm}
    \item Controls \ra Expressions \ra New, set name = densityGradient, type = VectorMeshVariable,
      definition = gradient(density)
    \item You can use the dropdown menus to accelerate typing (gradient
      will be found in Insert Function \ra Miscellaneous)
    \item Add a vector plot to show densityGradient, picking ``Uniformly located troughout the mesh'',
      displaying 5000 vectors, overlaying them onto a semi-transparent Pseudocolor plot of density
    \end{itemize}
  \end{beamerboxesrounded}
  \pause
  \begin{beamerboxesrounded}[upper=block head,lower=block body,shadow=true]{}
    \begin{itemize}\setlength{\itemsep}{0mm}
    \item Controls \ra Expressions \ra New, set name = truncatedDensity, type=VectorMeshVariable,
      definition=max(1,density)
    \item Now in the list of variables switch to truncatedDensity
    \end{itemize}
  \end{beamerboxesrounded}
\end{frame}

\begin{frame}{Pick}
  \begin{itemize}\setlength{\itemsep}{3mm}
  \item Interactively pick values inside the visualization window
  \item Load sineEnvelope.nc, visualize density with Pseudocolor, apply Operators \ra Selection \ra Clip,
    click Draw
  \item Data here is defined on nodes, not zones
    \pause
  \item Right click on the vis, select Mode \ra Zone Pick, and click anywhere on the vis -- it'll display
    8 nodes forming the zone, and their variable values
    \begin{itemize}\setlength{\itemsep}{0mm}
    \item each pick point leaves a marker that you can look up in the Pick window
    \item the Pick window displays information in tabs arranged by a point
    \end{itemize}
    \pause
  \item Similarly, Mode \ra Node Pick will display a single node, its variable, and its 8 ``incident''
    zones
  \end{itemize}
\end{frame}

\begin{frame}{Pick}
  \begin{itemize}\setlength{\itemsep}{2mm}
  \item Selecting Mode \ra Spreadsheet Pick shows a spreadsheet view of one of the dataset variables
    highlighting the picked node
    \pause
  \item Mode \ra Navigate will take you back to default interaction
  \item There are mode buttons in the toolbar of each vis window
    \pause
  \item Try loading a time-dependent dataset, e.g., abc
    \begin{itemize}\setlength{\itemsep}{0mm}
    \item a time slider will become active
    \item depending on the data use either Zone Pick or Node Pick
    \item inside the Pick window select ``Create time curve with next pick''
    \end{itemize}
  \end{itemize}
\end{frame}

\begin{frame}{Pick: lineout}
  \begin{itemize}\setlength{\itemsep}{2mm}
  \item Extracts 1D curves from 2D data (does not seem to work on 3D datasets)
  \item Load a 2D dataset or apply Operators \ra Slicing \ra Slice to a 3D dataset
  \item Select Mode \ra Lineout and draw a line -- the profile will be plotted in a new window
  \end{itemize}
\end{frame}

%% http://www.vtk.org/download - Download VTK textbook examples and data - PineRoot.tgz
%% http://www.vtk.org/files/release/7.0/VTKLargeData-7.0.0.zip

\begin{frame}{Pick: lineout}
  \begin{itemize}\setlength{\itemsep}{2mm}
  \item Controls \ra Query
  \item {\bf Option 1}: use Standard Queries
    \begin{itemize}\setlength{\itemsep}{0mm}
    \item very useful: Memory Usage
    \item quick ways to probe data: MinMax, NumNodes, NumZones, Average Value, Volume
    \item Lineout and Pick are also queries (have to enter selection manually)
    \item Certain queries provide a Time Curve button that calculates the query on each time step and
      creates a curve
    \end{itemize}
  \item {\bf Option 2}: use Python Query Editor for custom queries -- we'll cover it in the Scripting
    session
  \end{itemize}
\end{frame}

\begin{frame}{More on time sliders}
  \begin{itemize}\setlength{\itemsep}{2mm}
  \item Controls \ra Database correlations
  \end{itemize}
\end{frame}
