\mode<presentation>{
  \usetheme{Madrid}
  \setbeamertemplate{navigation symbols}{} % suppress nav bar
%\logo{\pichere{\put(-55,-5){\includegraphics[scale=0.1]{scinet-logo.png}}}}
\logo{\pichere{\put(-50,-8){\includegraphics[scale=0.45]{logos/hpcs2016}}}}
%\usecolortheme[named=RoyalBlue]{structure}
%\setbeamercolor*{alerted text}{fg=RoyalBlue}
\setbeamercolor*{alerted text}{fg=Blue}
}

\newcommand{\continue}{}


%\setbeamercovered{transparent}


% set bold tt and math fonts
\font\ttbf = cmbtt10
\font\ttit = cmitt10
\font\ttrm = cmtt10
\let\tt=\ttbf
\boldmath

%command for "some identifier": an italic tt type
%\newcommand{\some}[1]{{\ttit #1}}
\newcommand{\some}[1]{{\ttrm #1}}
\newcommand{\dollar}{\mbox{\tt\small\$}}

%colors in code: ways to color
%\newcommand{\ftype}[1]   {\textcolor{OliveGreen}{\tt #1}} %for types
%\newcommand{\fns}[1]     {\textcolor{MidnightBlue}{#1}}   %for namespaces
%\newcommand{\fprec}[1]   {\textcolor{cyan}{#1}}           %for preprocessor   
%\newcommand{\fintr}[1]   {\textcolor{RedViolet}{#1}}      %for intrinsic keywords
%\newcommand{\fcomm}[1]   {\textcolor{MidnightBlue}{#1}}   %for comments
%\newcommand{\literal}[1] {\textcolor{brown}{#1}}          %for literals
%\newcommand{\fstring}[1] {\literal{"#1"}}                 %for strings
%keywords
\newcommand{\cunsigned}  {\ftype{unsigned}}
\newcommand{\cint}       {\ftype{int }}
\newcommand{\cvoid}      {\ftype{void }}
\newcommand{\clong}      {\ftype{long }}
\newcommand{\cshort}     {\ftype{short }}
\newcommand{\cfloat}     {\ftype{float }}
\newcommand{\cdouble}    {\ftype{double }}
\newcommand{\cchar}         {\ftype{char }}
\newcommand{\creturn}       {\fintr{\tt return }}
\newcommand{\cstruct}       {\fintr{\tt struct }}
\newcommand{\cenum}         {\fintr{\tt enum }}
\newcommand{\cunion}        {\fintr{\tt union }}
\newcommand{\cclass}        {\fintr{\tt class }}
\newcommand{\ctypedef}      {\fintr{\tt typedef }}
\newcommand{\cif}           {\fintr{\tt if }}
\newcommand{\celse}         {\fintr{\tt else }}
\newcommand{\cfor}          {\fintr{\tt for }}
\newcommand{\cnew}          {\fintr{\tt new }}
\newcommand{\cdelete}       {\fintr{\tt delete }}
\newcommand{\cswitch}       {\fintr{\tt switch }}
\newcommand{\ccase}         {\fintr{\tt case }}
\newcommand{\cdefault}      {\fintr{\tt default}}
\newcommand{\cbreak}        {\fintr{\tt break}}
\newcommand{\cwhile}        {\fintr{\tt while }}
\newcommand{\cdo}           {\fintr{\tt do }}
\newcommand{\cfriend}       {\fintr{\tt friend }}
\newcommand{\cvirtual}      {\fintr{\tt virtual }}
\newcommand{\cprivate}[1]   {\fintr{\tt private#1}}
\newcommand{\cpublic}[1]    {\fintr{\tt public#1}}
\newcommand{\cprotected}[1] {\fintr{\tt protected#1}}
\newcommand{\cpoundincludebrackets}[1]{\fprec{\tt\#include \literal{<#1>}}}
\newcommand{\cpoundinclude}[1]{\fprec{\tt\#include \literal{"#1"}}}


%%%%%%%%%%%%%%%%%%%%%%%%%%%%%%%%%%%%%%%%%%%%%%%%%%%%%%%%%%%%%%%%%%%%%


% colors in code: ways to color
\newcommand{\ftype}[1]   {\textcolor{OliveGreen}{\tt #1}} % for types
\newcommand{\fns}[1]     {\textcolor{MidnightBlue}{#1}}   % for namespaces
\newcommand{\fprec}[1]   {\textcolor{BlueViolet}{#1}}     % for preprocessor   
\newcommand{\fintr}[1]   {\textcolor{RedViolet}{#1}}      % for intrinsic keywords
\newcommand{\fcomm}[1]   {\textcolor{MidnightBlue}{#1}}   % for comments
\newcommand{\literal}[1] {\textcolor{brown}{#1}}          % for literals
\newcommand{\fstring}[1] {\literal{"#1"}}                 % for strings

\newcommand{\fred}[1] {\textcolor{red}{#1}}
\newcommand{\forangered}[1]{\textcolor{OrangeRed}{#1}}
\newcommand{\fredorange}[1]{\textcolor{RedOrange}{#1}}
\newcommand{\frawsienna}[1]{\textcolor{RawSienna}{#1}}
\newcommand{\femerald}[1]{\textcolor{Emerald}{#1}}
\newcommand{\fbrickred}[1]{\textcolor{BrickRed}{#1}}
\newcommand{\fpinegreen}[1]{\textcolor{PineGreen}{#1}}
\newcommand{\fplum}[1]{\textcolor{Plum}{#1}}

%\input{auxs/shellnewcommands}
%\input{auxs/cppnewcommands}
%\input{auxs/lstsettings}


%a tabbed code environment in  box:
%sample:
%\begin{code}
%   int main() 
%  \begin{bracing}
%      int z = 0;
%      for (int i=0;i<60;i++)
%      \begin{tab}
%             z+=i;
%      \end{tab}
%   \end{bracing}
%\end{code}
\usepackage{setspace}
\xdefinecolor{headerbgcolor} {rgb}  {0.84, 0.84, 0.94}
\newlength{\tbl}\tbl0pc
\newenvironment{tab}
{\par\addtolength\tbl{1.2pc}\leftskip\tbl\tt}
{\addtolength\tbl{-1.2pc}\par\leftskip\tbl\rm}
\newenvironment{bracing}{{\scriptsize$\{$}\begin{tab}}{\end{tab}{\scriptsize$\}$}}

\makeatletter
\newenvironment{code}
{\begin{lrbox}{\@tempboxa}\begin{minipage}{0.92\columnwidth}\setstretch{0.85} \tt \setlength{\rightskip}{-0.4pc}}
{\end{minipage}\end{lrbox}\par\centerline{\colorbox{headerbgcolor}{\colorbox{white}{\usebox{\@tempboxa}}}}\vspace{1mm}}
\makeatother

%puts a zero-sized picture command at the current point
\newcommand{\pichere}[1]{\begin{picture}(0,0)\thicklines #1\end{picture}}

%a block for gotcha's
\newenvironment{gotcha}[1][]{\begin{block}{\color{Yellow}Gotcha: #1}}{\end{block}}

%\colorbox{#1}{\textcolor{#2}

\newcommand{\mediumblue}{\color{Blue}}
\newcommand{\black}{\color{Black}}
\newcommand{\restpointslide}[1]{
  \frame{
    \begin{quote}
      \Large\bf\mediumblue #1
    \end{quote}
  }
}

\newenvironment{examplesegment}{\setbeamercolor{frametitle}{bg=ForestGreen}
\setbeamercolor{block title}{bg=ForestGreen}
\setbeamercolor*{alerted text}{fg=Black}
\setbeamerfont{alerted text}{series=\bfseries}
\renewcommand{\mediumblue}{\color{ForestGreen}}}
{\setbeamercolor{frametitle}{bg=Blue}
\setbeamercolor{block title}{bg=Blue}
\setbeamercolor*{alerted text}{fg=Blue}
\setbeamerfont{alerted text}{series=\rmseries}
\renewcommand{\mediumblue}{\color{Blue}}}

\newenvironment{handsonsegment}{\setbeamercolor{frametitle}{bg=OrangeRed}
\setbeamercolor{block title}{bg=OrangeRed}
\setbeamercolor*{alerted text}{fg=Black}
\setbeamerfont{alerted text}{series=\bfseries}
\renewcommand{\mediumblue}{\color{OrangeRed}}}
{\setbeamercolor{frametitle}{bg=Blue}
\setbeamercolor{block title}{bg=Blue}
\setbeamercolor*{alerted text}{fg=Blue}
\setbeamerfont{alerted text}{series=\rmseries}
\renewcommand{\mediumblue}{\color{Blue}}}
